% ------------------------------------------------------
% Header
% ------------------------------------------------------

% Miles Smith
% MIT Mechanical Engineering Statement of Objectives

% Motivations for graduate school
% Topics that I want to study
% Relevant experience

% ------------------------------------------------------
% Preamble
% ------------------------------------------------------


\documentclass[12pt]{article}
\usepackage[compact]{titlesec}
\titleformat{\section}{\bfseries}{}{}{}[]
\usepackage{amsmath,amsthm,amssymb}
\usepackage{graphicx}
\usepackage{lipsum}
\usepackage{paralist}
\usepackage{fontspec}
\setmainfont{Times New Roman}
\usepackage{newtxmath}
\usepackage[font=footnotesize]{caption}
\usepackage[margin=1in]{geometry}
\usepackage[singlespacing]{setspace}
\pagenumbering{gobble}


\usepackage{cite}

\usepackage{natbib}
\setlength{\bibsep}{0pt plus 0.3ex}

% Math Packages
\usepackage{amsmath}
% Microtype (HUH?)
\usepackage[final]{microtype}
% Text Colors
\usepackage{xcolor}

% my custom commands
% ------------------
% note command
\newcommand{\note}[1]{\textsf{\textcolor{red}{#1}}}
% multi-line comment command
\newcommand{\comment}[1]{}


% ------------------------------------------------------
% Document
% ------------------------------------------------------

\begin{document}

\begin{flushright}
Miles Smith
\end{flushright}
\begin{center}
\large{\bf MIT Department of Mechanical Engineering\\}
\large{\it Statement of Objectives\\}
\hrulefill \\ 
\end{center}



\par
I am applying to Massachusetts Institute of Technology (MIT) to pursue a doctorate of philosophy (Ph.D.) in mechanical engineering to pursue my broad academic research interests in environmentally sustainable technology. More specifically, I am interested in basic energy science research in the domains of energy systems and the development of electrochemical energy storage technologies to enable large-scale adoption of intermittent renewable energy sources.  I am applying for a Ph.D. at MIT after completing a B.S. in mechanical engineering at the University of Maryland, Baltimore County (UMBC) and a M.S. in civil and environmental engineering at Stanford University.  Through graduate-level training in mechanical engineering at MIT I hope to develop skills as an engineer and scientist that will enable me to advance progress towards decarbonizing global energy systems and give myself a platform to support future scientists through mentorship and outreach.  Pursuing a Ph.D. in mechanical engineering will give me the skillset to operate and manage a research group in either a national laboratory or academic setting where I can innovate through research and create a culture to facilitate the success of others. 

\par
While at UMBC, my research primarily focused on energy harvesting systems. There, in Dr. Soobum Lee's Energy Harvesting and Design Optimization (ED) Lab, I developed a piezoelectric energy harvesting that could provide self-sustaining power for sensors in wind turbine blades by harnessing the rotational energy of a wind turbine. For the design of this, my idea was that, as the turbine blade rotates, using a ball bearing connected to a weighted-rotating disk there should be some sort of oscillatory motion as the disk swings back and forth (ideally like a pendulum of sorts). Thus, fixing the piezoelectric transducer (PZT) like a cantilever beam within the energy harvesting system and connecting repelling magnets to the piezoelectric material and the rotating disk, there should be some oscillatory strain on the PZT and a useful power output. This research served as a foundation for my broad interests in environmentally sustainable technology and renewable energy systems. 

\par
Since being at Stanford, I have conducted research in sustainable energy systems and energy storage through collaborative research opportunities with faculty and external organizations.  Through the Stanford Energy Club, I led a project team of four Stanford students in conjunction with a team from EDF Innovation Lab to estimate the economic impact of increased integration of bundled PV and energy storage systems on the electric grid.  Following this project, I accepted a summer position at MIT Lincoln Laboratory in the Energy Systems group led by Erik Limpaecher and supervised by Dr. Theodore Bloomstein as GEM Fellowship intern.  While at MIT Lincoln Laboratory, I developed a control system using a simple circuit and Arduino to characterize the state-of-health of a novel battery architecture being developed by the group using low-cost and non-invasive methods.  Connecting the battery in series with a resistor and the Arduino, I found that I was able to measure the voltage and current through the battery to charge and discharge the battery. Then, by probing the battery with sinusoidal, rectangular, and triangular voltage waveforms and measuring the change in voltage through the battery and applying a discrete Fourier transform to the battery voltage, I was able to estimate the equivalent series resistance (ESR) of the battery, which can be used as an indicator of cell degradation by comparing the ESR at specific state-of-charge over multiple charge and discharge cycles.  Further, as a GEM Fellow, I have the opportunity to return to Lincoln Laboratory this upcoming summer and continue this project.  For my second summer at Lincoln Laboratory, I am hoping to focus on three different challenges for the system that could still be improved -- (1) increasing measurement accuracy by amplifying the Arduino to probe the battery at higher frequencies and using a higher resolution controller to enable more precise measurement, (2) the integration of an adaptive extended Kalman filter to measure the state-of-charge of the battery and provide further validation of the ESR, and (3) validate the system through repeated cycle testing. The implication of this research would be a novel way to characterize the state-of-health of a battery using small-scale microcontrollers that could be integrated within a battery cell to measure the state of health indicators of a battery with more accurate estimations of the state variables. 

\par
This project also served as the conceptual foundation for a project that I began this quarter in Prof. Simona Onori's lab at Stanford where we are analyzing battery field data to develop machine learning and phenomenological models to characterize battery state-of-health in non-laboratory situations. Then, I hope to apply my skills developed here studying ``big data" and machine learning techniques as they pertain to batteries and model development to my project at Lincoln Laboratory when I return this summer to improve the accuracy of state-of-health estimations from data collected by the microcontroller. 

\par
As this pertains to a Ph.D., I think my research background in electrochemical control systems and battery state-of-health estimation compliments the research of Prof. Betar Gallant and Prof. Yang Shao-Horn, both of whom conduct experimental research developing novel battery chemistries. Also, within the material science department, Prof. Yet-Ming Chiang stands out as another faculty member that I would love to collaborate with.  Until now, most of my battery research has focused on control systems and has been derived from an electrical engineering approach, but, for my Ph.D., I am hoping to combine my background in electrochemical control systems with a deeper understanding of battery chemistry and materials science through joint research with my group at Lincoln Laboratory and the faculty at MIT.  From there, I hope to apply my background in batteries and control systems to develop electrochemical devices that can be integrated in grid-scale applications to improve the reliability of an electric grid that is highly dependent on renewables. 


\end{document}



% ------------------------------------------------------
% End of Document
% ------------------------------------------------------
